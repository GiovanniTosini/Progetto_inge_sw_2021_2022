\underline{}\documentclass{article}

\usepackage[T1]{fontenc}
\usepackage[utf8]{inputenc}
\usepackage[italian]{babel}
\usepackage[colorlinks]{hyperref}

\title{Documentazione del progetto di \\ Ingegneria del Software}
\author{Gabriele Faustinoni \\ Alessandro Lorini \\ Giovanni Tosini}
\date{}

\begin{document}

    \begin{titlepage}
        \maketitle
    \end{titlepage}

    \newpage
    \tableofcontents
    \newpage

    \section{Note introduttive}

    Il sistema in questione è stato creato per un'azienda interinale 
    per la gestione dei lavoratori stagionali.

    \section{Requisiti}

    Gli operatori dell'agenzia per poter lavorare avranno necessità di essere in possesso delle loro credenziali, le quali verranno create appositamente dagli sviluppatori del suddetto sistema.

    I lavoratori stagionali interessati potranno presentarsi agli sportelli dell'agenzia per farsi registrare, se già presenti in archivio o per aggiornare il loro storico lavorativo degli ultimi 5 anni.
    La registrazione consiste nel salvataggio dei dati del lavoratore

    \begin{itemize}
        \item nome
        \item cognome
        \item luogo e data di nascita
        \item nazionalità
        \item indirizzo di residenza
        \item telefono cellulare (non necessariamente obbligatorio)
        \item lingue parlate
        \item patente e se automunito
        \item eventuali esperienze precedenti (non sono necessariamente i lavori fatti negli ultimi 5 anni)
        \item periodi e zone di disponibilità per lavorare
        \item una persona da contattare per eventuali emergenze della quale saranno necessari: 
        \begin{itemize}
            \item nome
            \item cognome
            \item indirizzo e-mail
            \item numero di telefono
        \end{itemize}
    \end{itemize}

    L'aggiornamento del loro storico lavorativo consiste nella semplice ricerca del lavoratore (se presente in archivio) tramite nome, cognome e data di nascita e l'inserimento di:

    \begin{itemize}
        \item periodo in cui ha lavorato
        \item nome azienda 
        \item luogo di lavoro
        \item retribuzione lorda giornaliera
        \item mansioni svolte
    \end{itemize}

    Gli operatori dell'agenzia inoltre possono usare una funzione di ricerca che permetterà di ricercare una figura lavorativa specifica (usando l'inclusione obbligatoria di tutti i parametri) oppure di una figura con solo alcuni dei criteri necessari, una volta fatta la ricerca l'elenco verrà stampato in un'apposita finestra. I parametri della funzione ricerca sono:

    \begin{itemize}
        \item nome
        \item cognome
        \item periodo di disponibilità
        \item città di residenza
        \item patente e se automunito
        \item mansioni
        \item lingue parlate
    \end{itemize}

    \section{Scenari d'uso}

    TODO sono da controllare e rifare per bene eventualmentehgvb

\end{document}


